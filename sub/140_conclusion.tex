% -*- coding: utf-8 -*-
% !TEX root = ../main.tex

\subsection{Conclusion}\label{ssec:conclusion_conclusion}

Ce stage m'as permis de faire le tour de plusieurs points clés du fonctionnement de \verb!PALLAS!.
En effet, j'ai pu, dans un premier temps, afin de prendre en main les outils nouveaux, étudier le fonctionnement d'une application très utile en pratique, \verb!pallas\_print!.
Dans un second temps, j'ai eu pu étudier l'impact général du tracé avec \verb!EZTrace! au format \verb!PALLAS! sur le programme.
Enfin, j'ai vérifié des points clés de l'écriture sur les disques avec une étude de la tailles des sous-vecteurs ainsi que de la compression des données en étudiant différents algorithmes 
de compression proposés par \verb!PALLAS!, tant sous l'aspect de la perte d'information que sur leurs impacts sur le temps total d'exécution du programme.

%%%%%%%%%
%%%%%%%%%
%%%%%%%%%
%%%%%%%%%

\subsection{Retours d'expérience}\label{ssec:conclusion_retours}

% TODO Vos retours d'expérience : ce que vous avez appris/découvert

A travers ce stage j'ai pu mettre en application beacoup des outils appris lors de ma première année à l'école, comme la programmation impérative et orientée objet, ainsi que 
l'utilisation du \verb!shell! et le scripting avec \verb!bash! qui m'as permis d'automatiser beaucoup de tâches ainsi que de récupérer aisément différentes données liées 
à l'exécution de différents programmes.
De plus, j'ai beaucoup appris à travers l'utilisation de programmes nouveaux comme \verb!screen! qui permet de lancer des programmes et pouvoir se "détacher"
du processus concerné. De plus, j'ai appris l'utilisation de la bibliothèque \verb!time! qui permet d'avoir très facilement des données liées au temps dans un programme. Côté, \verb!python!
j'ai appris à utiliser différentes bibliothèques comme par exemple \verb!matplotlib! ou \verb!pandas! pour récupérer et aggréger des données puis les visualiser.

De plus, ce stage m'as permis de découvrir le monde de la recherche. En effet, j'ai pu apprendre à travailler en autonomie sur des sujets demandant des prises d'initiative, tout en étant 
encadré par des enseignants qui ont su me guider au mieux tout au long de ce stage, avec 
parfois le besoin de mettre en place de nouveaux outils (comme avec \verb!pallas\_durations!). Le cadre de \verb!Télécom SudParis!, quant à lui, m'as permis d'échanger au quotidien 
avec des enseignants-chercheurs ainsi que des doctorants et en apprendre plus sur leur travail. Enfin, j'ai eu l'occasion d'assister à une soutennace de thèse, ce qui m'as 
donné un apperçu de la finalité du travail des doctorants que j'ai côtoyé.

%%%%%%%%%
%%%%%%%%%
%%%%%%%%%
%%%%%%%%%

\subsection{Perspectives}\label{ssec:conclusion_perspectives}

% TODO Les perspectives d'amélioration ou d'utilisation de votre programme/de ce que vous avez développé

Tout d'abord, j'aurais apprécié avoir le temps de mettre en place et d'évaluer le flush progressif sur les disques des différents sous-vecteurs pendant l'exécution, et non d'un seul coup à la fin de celle-ci.\\
De plus, il serait intéressant de fixer le bug à cause duquel il n'est pas possible de parcourir deux fois les données d'une trace (l'équipe avait déjà reglé le fait qu'il était impossible 
d'ouvrir deux traces dans un même programme, ce qui pouvait être problématique pour de futures analyses de traces).

%%%%%%%%%
%%%%%%%%%
%%%%%%%%%
%%%%%%%%%
