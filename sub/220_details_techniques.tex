% -*- coding: utf-8 -*-
% !TEX root = ../main.tex

% TODO Expliquer la solution implémentée plus en détails (extraits de code, interfaces des applications, l’interface graphique des outils utilisés...)


\subsection{Structures de données mises en place}\label{ssec:details_durations}

Afin de mettre en place les outilss utiles à l'analyse des programme, une structure de données \verb!duration! a été mise en place.
Celle-ci permet d'avoir des données statitsiques comme le temps total, les durées minimale, maximale, le nombre d'appels, mais également un tableau contenant les durées pour chaque appel et les tailles associées
(utile pour observer le temps d'appel à une fonction en fonction de la taille du bloc en question).

\tcbFileInput{cpp}{\faCode}{lst/duration.cpp}{Structure duration}{lst:duration}

Ensuite, il a fallu mettre en place la fonction \verb!update_durations! qui permet de mettre à jour la structure \verb!duration!. Cette fonction permet aussi d'initialiser la structure
lors du premier appel.
\tcbFileInput{cpp}{\faCode}{lst/struct.cpp}{Struct Timespec}{lst:update}
Les \verb!struct timespec! sont des structures liées à la librairie \verb!time.h! qui permet de stocker le temps sous forme d'une composante en secondes et d'une composante en nanosecondes, 
ce qui est utile ici dans le cadre d'une analyse de fonctions de les appels peuvent être extrêmement courts.

\tcbFileInput{cpp}{\faCode}{lst/update.cpp}{Fonction de mise à jour de la structure}{lst:update}

A la fin du code (à la fin de la dernière fonction appellée par la librairie \verb!PALLAS! qui est une fonction qui permet de finaliser l'écriture de la trace sur le disque), 
on place les deux fonctions suivantes en fonction du besoin. La première enregistre au format
\verb!.csv! les données statistiques d'une structure \verb!duration! et l'autre toutes les données de chaque appel.
De plus, on utilise un tableau global qui nous permet d'avoir les ressources accessibles partout dans le code.

\tcbFileInput{cpp}{\faCode}{lst/write.cpp}{Fonctions d'enregistrement de la structure}{lst:write}

\subsection{Analyse de la compression - test $R^2$}\label{ssec:comp}

Un test statistique classique pour comparer deux échantillons de tailles identiques est le test $R^2$.\\
Voici la formule utilisée, avec $y$ l'échantillon de référence et $\hat{y}$ l'échantillon à comparer :

\[
R^2 \;=\; 1 - \frac{SS_{\mathrm{res}}}{SS_{\mathrm{tot}}}
\qquad\text{où}\qquad
SS_{\mathrm{res}} \;=\; \sum_{i=1}^{n} (y_i - \hat{y}_i)^2,
\;\;
SS_{\mathrm{tot}} \;=\; \sum_{i=1}^{n} (y_i - \bar{y})^2.
\]

\[
\bar{y} \;=\; \frac{1}{n}\sum_{i=1}^{n} y_i\]


Ensuite, l'implémentation est réalisée de la manière suivante, qui permet de passer directement les adresses des fichiers \verb!.csv! contenant les timestamps :

\tcbFileInput{cpp}{\faCode}{lst/r2.cpp}{Mise en place du test $R^2$}{lst:r2}

Par ailleurs, pour parcourir la trace, on procède en parcourant les archives composant la trace, puis les threads relatifs à ces archives puis tous les évènements de ces threads.

\tcbFileInput{cpp}{\faCode}{lst/loop.cpp}{Parcours de la trace}{lst:trace}
