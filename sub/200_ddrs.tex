% -*- coding: utf-8 -*-
% !TEX root = ../main.tex

% TODO Annexe DDRS


Mon stage s’est déroulé au laboratoire \verb!SAMOVAR! de \verb!Télécom SudParis! (Institut Mines–Télécom), au sein de l’équipe \verb!Benagil! \cite{benagil}.
\verb!SAMOVAR! est fortement inséré dans l’écosystème de l’Institut Polytechnique de Paris (IP~Paris), notamment via les centres interdisciplinaires E4C — Energy4Climate et Hi!~PARIS.

\subsection{Responsabilité sociétale}
Les pratiques de l’unité sont conformes aux directives des tutelles (IMT/TSP) en matière de gestion RH, éthique, sécurité, environnement et protection des données. 
L’unité dispose d’un référent DDRS et participe à des actions de formation et d’analyse d’empreinte carbone à TSP \cite{hceres}. 
TSP a nommé un DPO (délégué à la protection des données) et publie ses informations légales (RGPD - Réglement Général de la Protection des Données) \cite{legal}. l’IMT encadre les usages par une Charte informatique rattachée à la PSSI.

Au niveau Groupe, l’IMT déploie (2024–2026) un plan Égalité femmes/hommes et un plan de lutte contre le harcèlement, les violences sexistes et sexuelles et les discriminations (HVSSD)\cite{hf}.
À TSP, une cellule d’écoute et des actions de prévention sont actives (écoute, accompagnement, sensibilisation) \cite{prev}.
Le rapport Hcéres pointe toutefois un point de vigilance : une sous-représentation féminine dans la direction de SAMOVAR sur la période évaluée ; 
la Direction de TSP a annoncé (févr. 2025) des mesures de correction lors du renouvellement 2025 \cite{hceres}.

Le Hcéres recommande d’appliquer plus effectivement la politique de science ouverte (dépôts HAL, données FAIR, logiciels ouverts) \cite{hceres}.
En 2025, l’IMT publie une charte de science ouverte et affiche 74\% d’articles en accès ouvert (édition 2024 du Baromètre, publiée en janv.~2025)

Le Hcéres souligne des relations soutenues avec le monde économique (sujets de souveraineté numérique, santé) ; environ un tiers des thèses sont Cifre (Conventions industrielles de formation par la recherche), 
et des laboratoires communs industriels sont actifs \cite{hceres}.

\subsection*{Développement durable}

En 2024, l’IMT a réalisé un bilan carbone (données~2022) et mis en place un plan de transition visant \~25\% d’émissions d’ici~2027; la démarche 
s’accompagne d’actions «\,\emph{éco-campus}\,» dans chaque école \cite{transition}.

Sur le campus d’Évry, TSP a conduit en 2024 des travaux de rénovation énergétique (\emph{France Relance}) portant notamment sur le gymnase (isolation, pompe à chaleur, éclairage) et les menuiseries des résidences U1–U2 \cite{renov}.

En 2025, TSP renforce le tronc commun avec >200~h dédiées à la transition écologique (8~ECTS), une Charte de l’ingénieur du numérique responsable et la chaire d’enseignement INTEGRATE \cite{ddrs}.




Ainsi Télécom SusParis possède un cadre institutionnel robuste avec les référents DDRS, DPO et les plans pour l'égalité et contre les HVSSD.\\
De plus, elle forme tous ses élèves aux enjeux du numérique responsable et est prête à se renouveller à travers différents travaux réalisés sur le campus.\\
Il faut toutefois faire attention à la mixité dans l'équipe de direction.
